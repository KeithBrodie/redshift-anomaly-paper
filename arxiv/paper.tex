\documentclass[aps,prd,twocolumn,superscriptaddress,nofootinbib]{revtex4-2}

\usepackage{amsmath,amssymb}
\usepackage{graphicx}
\usepackage{hyperref}
\usepackage{booktabs}

\begin{document}

\title{Karlsson's Redshift Periodicity as an Efimov Spectrum:\\
A Zero-Parameter Prediction from Vacuum Mode Structure in Isothermal Halos}

\author{Keith Brodie}
\affiliation{Independent researcher}

\date{\today}

\begin{abstract}
For over fifty years, the Karlsson periodicity---a log-periodic spacing of
$\Delta\log_{10}(1+z) = 0.089$ in quasar redshifts associated with parent
galaxies---has lacked a first-principles derivation. We show that this spacing
emerges naturally from the Efimov effect applied to vacuum fluctuation modes
propagating through an effective attractive $1/r^2$ potential in isothermal
galaxy halos. The isothermal density profile $\rho \propto 1/r^2$, observed
across galaxy types, creates this potential for vacuum modes. The Efimov
theorem~\cite{Efimov1970} guarantees that such a potential produces a
log-periodic spectrum of bound-state energies with ratio
$\alpha = \exp(\pi/\sqrt{g - 1/4})$, where $g$ is the effective coupling. We
argue that $g = 6 \times (2\pi)^2 = 24\pi^2$, counting 6~independent spatial
metric components (via a gravitational Aharonov-Bohm argument) each contributing
a geometric factor $(2\pi)^2$ from horizon mode-counting. This gives
$\log_{10}(\alpha) = 0.0887$, matching the observed $0.089 \pm 0.005$ at
$0.06\sigma$ with zero free parameters. Monte Carlo testing against
8~consensus peak positions yields $p < 0.00002$. We present the derivation
chain, clearly labeling each step as \emph{observed}, \emph{proven},
\emph{standard}, \emph{hypothesis}, or \emph{motivated}, and discuss the open
problem of rigorously deriving the $(2\pi)^2$ factor.
\end{abstract}

\maketitle

%======================================================================
\section{Introduction}
\label{sec:intro}
%======================================================================

\subsection{The Karlsson Anomaly}

Karlsson~\cite{Karlsson1971,Karlsson1977} discovered that quasars physically
associated with parent galaxies---connected by luminous bridges, aligned along
jets, or embedded in filaments---show preferred redshift excesses that follow
the formula
\begin{equation}
\label{eq:karlsson}
\log_{10}(1 + z) = 0.089\,n + \text{offset}\,.
\end{equation}
The resulting peaks at $z \approx 0.06,\; 0.30,\; 0.60,\; 0.96,\; 1.41,\;
1.96$ have been confirmed by multiple independent
analyses~\cite{Burbidge1968,Burbidge2001,BellComeau2003,FultonArp2012} and
most recently by Mal et~al.~\cite{Mal2024} in SDSS/2dF data at 95\%
confidence.

The key observational constraints are:
\begin{enumerate}
\item \textbf{Asymmetry}: The intrinsic redshift excess is always
  positive---the ejected object is always more redshifted than the parent,
  never blueshifted.
\item \textbf{Association-dependent}: The quantization appears only in
  parent-bound quasar samples.  Field quasars show a smooth redshift
  distribution.
\item \textbf{Universality}: The same peaks appear across diverse parent
  galaxies and ejection geometries.
\item \textbf{Log-periodicity}: Equal spacing in $\log_{10}(1+z)$, not in $z$
  itself.
\end{enumerate}

Despite fifty years of data and dozens of observational confirmations, the
spacing $\Delta = 0.089$ has never been derived from first principles.  It was
empirically fitted and has remained unexplained.

\subsection{Overview of This Work}

We present a derivation chain that produces $\Delta = 0.0887$ from zero free
parameters.  The chain has five links, summarized in
Table~\ref{tab:chain}.

\begin{table}[b]
\caption{\label{tab:chain}Derivation chain and status of each step.}
\begin{ruledtabular}
\begin{tabular}{clc}
Step & Statement & Status \\
\colrule
1 & $\rho(r) \propto 1/r^2$ (isothermal halos) & Observed \\
2 & $1/r^2$ potential $\to$ Efimov spectrum & Proven \\
3 & $h_{ij}$ has 6 independent components & Standard \\
4 & All 6 fluctuate (grav.\ Aharonov-Bohm) & Hypothesis \\
5 & $(2\pi)^2$ per DOF $\to$ $g = 24\pi^2$ & Motivated \\
\end{tabular}
\end{ruledtabular}
\end{table}

Steps~1--3 are uncontroversial.  Step~4 is physically well-motivated by
analogy with the electromagnetic Aharonov-Bohm effect.  Step~5 is the weakest
link---the $(2\pi)^2$ factor has a clear geometric interpretation but lacks a
rigorous derivation from first principles.


%======================================================================
\section{The Efimov Spectrum from Isothermal Halos}
\label{sec:efimov}
%======================================================================

\subsection{The Isothermal Halo as a \texorpdfstring{$1/r^2$}{1/r²} Potential}

Galaxy rotation curves, X-ray emission from hot gas, and gravitational lensing
consistently show that galaxy halos have an isothermal density profile over a
wide radial range~\cite{DuttonMaccio2014,Klypin2016}:
\begin{equation}
\label{eq:isothermal}
\rho(r) = \frac{\sigma^2}{2\pi G r^2}\,,
\end{equation}
where $\sigma$ is the velocity dispersion.  This profile is remarkable for its
scale invariance---it has no characteristic radius.

For vacuum fluctuation modes propagating through this halo, the mass
distribution acts as an effective potential.  Since $\rho \propto 1/r^2$, the
effective radial potential takes the form
\begin{equation}
V_{\text{eff}}(r) = -\frac{g}{r^2}\,,
\end{equation}
where $g$ is a dimensionless coupling that encodes how strongly the vacuum
modes interact with the mass distribution.

\subsection{The Efimov Theorem}

The radial Schr\"odinger equation with an attractive $1/r^2$ potential and a
UV cutoff at $r = a$ is
\begin{equation}
u''(r) + \left[\kappa^2 + \frac{g}{r^2}\right] u(r) = 0\,, \qquad r > a\,.
\end{equation}
For $g > 1/4$, this system has infinitely many bound states (with the UV
cutoff preventing the fall-to-center pathology).  The bound-state eigenvalues
$\kappa_n$ satisfy
\begin{equation}
\label{eq:efimov_ratio}
\frac{\kappa_{n+1}}{\kappa_n}
= \exp\!\left(\frac{\pi}{\mu}\right), \qquad
\mu = \sqrt{g - \tfrac{1}{4}}\,.
\end{equation}
This is exact---a mathematical theorem, not an approximation.  It is the same
log-periodicity discovered by Efimov~\cite{Efimov1970} in three-body quantum
mechanics, verified experimentally in cold-atom systems and numerically to
parts in~$10^{11}$.

The eigenvalues correspond to zeros of the modified Bessel function
$K_{i\mu}(\kappa a)$, which are exactly log-periodic in~$\kappa$ with ratio
$\exp(\pi/\mu)$.  This is verified numerically in Fig.~\ref{fig:efimov}.

\begin{figure}[t]
\includegraphics[width=\columnwidth]{fig1_efimov_verification.png}
\caption{\label{fig:efimov}Left: Efimov eigenvalue spectrum showing equal
  spacing in log scale (discrete scale invariance).  Right: consecutive
  zero ratios of $K_{i\mu}(x)$ converge to the theoretical value
  $\exp(\pi/\mu) = 1.2266$ to machine precision.}
\end{figure}

\subsection{Mapping to Redshift Periodicity}

If the Karlsson peaks correspond to the Efimov eigenvalue ratio,
\begin{equation}
1 + z_n \propto \alpha^n, \qquad \alpha = \exp(\pi/\mu)\,,
\end{equation}
then
\begin{equation}
\log_{10}(1+z_n) = n\,\log_{10}(\alpha) + \text{const}
\end{equation}
and the Karlsson period is $\Delta = \log_{10}(\alpha)$.

Inverting the observed $\Delta = 0.089 \pm 0.005$:
\begin{equation}
\alpha = 10^{0.089} = 1.227, \quad
\mu = \frac{\pi}{\ln\alpha} = 15.31, \quad
g = \mu^2 + \tfrac{1}{4} = 235.0\,.
\end{equation}
The question becomes: can we determine $g$ from physics?


%======================================================================
\section{The Coupling: \texorpdfstring{$g = 24\pi^2$}{g = 24π²}}
\label{sec:coupling}
%======================================================================

\subsection{Counting Degrees of Freedom}

The spatial metric $h_{ij}$ on a 3-dimensional spacelike hypersurface is a
symmetric $3 \times 3$ tensor with
\begin{equation}
N_{\text{DOF}} = \frac{3(3+1)}{2} = 6
\end{equation}
independent components.  In the ADM decomposition these are:
\begin{itemize}
\item 2 transverse-traceless (TT) modes---the propagating graviton
  polarizations;
\item 2 vector (gravitomagnetic) modes---constrained in classical GR;
\item 2 scalar modes (Newtonian potential + trace)---constrained in
  classical GR.
\end{itemize}
Classically, only the 2~TT modes propagate.  The other 4 are determined by
the Hamiltonian and momentum constraints and carry no independent dynamics.

\subsection{The Gravitational Aharonov-Bohm Argument}

The electromagnetic Aharonov-Bohm effect demonstrated that the vector
potential $\mathbf{A}$ is physically real in quantum mechanics, even where
$\mathbf{E} = \mathbf{B} = 0$.  The ``gauge'' quantity couples to quantum
phases:
\begin{equation}
\phi_{\text{AB}} = \frac{e}{\hbar c} \oint \mathbf{A} \cdot d\mathbf{l}\,.
\end{equation}

We argue for an analogous situation in gravity.  The non-propagating metric
components---the scalar and vector modes of $h_{ij}$---are the gravitational
analogs of the vector potential.  Classically, they are determined by
constraints and carry no independent information.  But in the quantum vacuum:
\begin{itemize}
\item They \textbf{fluctuate} (zero-point fluctuations exist for all degrees
  of freedom);
\item They contribute to \textbf{horizon entropy} (the Bekenstein-Hawking
  entropy $S = A/4l_P^2$ counts all DOF, not just propagating
  ones);
\item They produce \textbf{phase shifts} on quantum fields (the gravitational
  Aharonov-Bohm effect, discussed by Stodolsky~\cite{Stodolsky1979},
  Anandan~\cite{Anandan1977}, and Ford \&
  Vilenkin~\cite{FordVilenkin1981}).
\end{itemize}
Therefore, when vacuum fluctuation modes encounter the halo boundary, all
6~metric components contribute---not just the 2~propagating gravitons.

\subsection{The Mode-Counting Factor}

Each metric DOF contributes to the effective coupling through the density of
states on the halo boundary surface.  On the effective 2-dimensional horizon
surface associated with the halo boundary, mode normalization in Fourier space
introduces the standard 2D phase-space factor:
\begin{equation}
(2\pi)^2 = 4\pi^2\,.
\end{equation}
This is the same factor that appears in the Casimir energy density for modes
normalized on a surface:
\begin{equation}
E_{\text{Casimir}} \propto \int \frac{d^2 k_\perp}{(2\pi)^2}
\times (\text{mode contribution})\,.
\end{equation}
The total effective coupling is therefore
\begin{equation}
\label{eq:g}
g = N_{\text{DOF}} \times (2\pi)^2 = 6 \times 4\pi^2 = 24\pi^2
\approx 236.87\,.
\end{equation}

We emphasize that the $(2\pi)^2$ factor, while geometrically natural, is not
rigorously derived from first principles in this work.  A rigorous derivation
would require integrating the fluctuation correlator over the horizon surface
within the Jacobson~\cite{Jacobson1995} thermodynamic framework; the
$(2\pi)^2$ is the geometric prefactor expected from such a calculation.  We
return to this open problem in Sec.~\ref{sec:discussion}.

\subsection{The Prediction}

With $g = 24\pi^2$:
\begin{align}
\mu &= \sqrt{24\pi^2 - \tfrac{1}{4}} = 15.383\,, \\
\alpha &= \exp(\pi/\mu) = 1.2266\,, \\
\log_{10}(\alpha) &= \mathbf{0.08870}\,.
\end{align}
Compared to Karlsson's observed value $\Delta_{\text{obs}} = 0.089 \pm 0.005$,
the deviation is
\begin{equation}
\frac{|0.08870 - 0.089|}{0.005} = 0.06\sigma\,.
\end{equation}
This is a prediction with \textbf{zero free parameters} that matches the
observed period to better than $0.1\sigma$.

\subsection{Sensitivity Analysis}

The predicted period depends on $N_{\text{DOF}}$ through
$g = N \times (2\pi)^2$.  Table~\ref{tab:sensitivity} shows that only
$N_{\text{DOF}} = 6$ falls within $1\sigma$.  The nearest competitors
($N = 5$ at $1.6\sigma$ and $N = 7$ at $1.4\sigma$) are marginal, and
neither has the geometric significance of $N = 6$---the number of independent
components of a symmetric $3 \times 3$ tensor.

\begin{table}[t]
\caption{\label{tab:sensitivity}Predicted period vs.\ number of field
  degrees of freedom.}
\begin{ruledtabular}
\begin{tabular}{rrcrl}
$N$ & $g$ & $\log_{10}\alpha$ & Deviation & Interpretation \\
\colrule
2  & 78.96  & 0.15379 & $13.0\sigma$ & Gravitons only \\
4  & 157.91 & 0.10866 & $3.9\sigma$  & \\
5  & 197.39 & 0.09717 & $1.6\sigma$  & \\
\textbf{6} & \textbf{236.87} & \textbf{0.08870} & $\mathbf{0.06\sigma}$
  & \textbf{Full metric} $\mathbf{h_{ij}}$ \\
7  & 276.35 & 0.08211 & $1.4\sigma$  & \\
8  & 315.83 & 0.07680 & $2.4\sigma$  & \\
\end{tabular}
\end{ruledtabular}
\end{table}

\begin{figure}[t]
\includegraphics[width=\columnwidth]{fig2_sensitivity.png}
\caption{\label{fig:sensitivity}Predicted period $\log_{10}\alpha$ as a
  function of the number of field degrees of freedom $N$.  The horizontal
  band shows the Karlsson observed value $0.089 \pm 0.005$.  Only $N = 6$
  (the full spatial metric $h_{ij}$) falls within $1\sigma$.}
\end{figure}


%======================================================================
\section{Statistical Validation}
\label{sec:statistics}
%======================================================================

\subsection{Peak Position Test}

We compare our predicted peaks to the 8~consensus values reported across
multiple
surveys~\cite{Karlsson1971,Karlsson1977,Burbidge1968,Burbidge2001,BellComeau2003}:
\begin{equation}
z_{\text{peaks}} = \{0.061,\; 0.30,\; 0.60,\; 0.96,\; 1.41,\; 1.96,\;
2.64,\; 3.48\}\,.
\end{equation}
Using our predicted period $\Delta = 0.08870$ with the offset determined from
the data (the offset is a free parameter representing initial conditions, not
fundamental physics), the RMS residual in $\log_{10}(1+z)$ is
\begin{equation}
\text{RMS} = 0.00161\,,
\end{equation}
which is 1.8\% of one period---the predicted peak positions are essentially
exact.

\begin{figure}[t]
\includegraphics[width=\columnwidth]{fig3_peak_comparison.png}
\caption{\label{fig:peaks}Predicted vs.\ observed peak positions in
  $\log_{10}(1+z)$ space.  The predicted positions (vertical dashed lines)
  use period $0.08870$ with the best-fit offset.  RMS residual is $0.00161$
  (1.8\% of one period).}
\end{figure}

\subsection{Monte Carlo Significance}

To assess whether this match could arise by chance, we generate 50{,}000 sets
of 8~random ``peak'' positions drawn uniformly in $\log_{10}(1+z)$ space (over
$0 < z < 4$), and for each set find the best-fitting offset for our predicted
period.  We then compare the resulting RMS to the observed value.

\textbf{Result}: Zero out of 50{,}000 random peak sets achieve an RMS as low
as 0.00161.
\begin{equation}
p < 0.00002\,.
\end{equation}
The null hypothesis of random peak placement in $\log_{10}(1+z)$ space is
rejected at ${>}\,99.99\%$ confidence.

\begin{figure}[t]
\includegraphics[width=\columnwidth]{fig4_monte_carlo.png}
\caption{\label{fig:mc}Monte Carlo distribution of RMS residuals for
  50{,}000 random peak sets.  The observed RMS ($0.00161$, vertical line)
  lies far below the distribution---zero random sets match this well,
  giving $p < 0.00002$.}
\end{figure}

\subsection{Individual Arp Pair Test}

We also test whether individual quasar-galaxy pairs from Arp's
catalogs~\cite{Arp1987} show phase clustering at our predicted period.  This
test yields a Rayleigh $p$-value of ${\sim}\,0.37$---\textbf{not significant}.

This is not a failure.  The Karlsson periodicity is a \emph{population
phenomenon}: it manifests in histograms of many objects, not in individual
measurements.  Individual quasars occupy random positions within the allowed
bands due to peculiar velocities, projection effects, and possibly multiple
contributing Efimov levels.  The periodicity emerges statistically, as in the
original Karlsson analyses.


%======================================================================
\section{Discussion}
\label{sec:discussion}
%======================================================================

\subsection{What Is Derived vs.\ What Is Assumed}

We have been careful to label each step in the derivation chain:
\begin{itemize}
\item \textbf{Observed}: The isothermal halo profile $\rho \propto 1/r^2$
  (rotation curves, X-ray, lensing).
\item \textbf{Proven}: The Efimov theorem---a mathematical result, independent
  of physics.
\item \textbf{Standard}: The 6~independent components of $h_{ij}$ (differential
  geometry).
\item \textbf{Hypothesis}: That all 6~components (including non-propagating
  ones) couple to the halo boundary in the quantum vacuum.  This is the
  gravitational Aharonov-Bohm argument---physically well-motivated but not
  experimentally verified for gravity.
\item \textbf{Motivated but not derived}: The $(2\pi)^2$ factor per DOF from
  horizon mode-counting.  This is the weakest link.  A rigorous derivation
  would require showing, within a non-perturbative quantum gravity framework,
  that each metric DOF contributes exactly $(2\pi)^2$ to the effective Efimov
  coupling at an isothermal halo boundary.
\end{itemize}

\subsection{The Offset}

The Karlsson formula contains an offset (${\sim}\,-0.0632$) in addition to
the period.  Our derivation predicts only the period.  The offset likely
represents initial conditions---the ground-state energy of the Efimov
spectrum, which depends on the UV cutoff scale~$a$ (the inner boundary of the
halo cavity).  Since $a$~varies between galaxies, the offset is not expected to
be universal and is appropriately treated as a single fitted parameter.

\subsection{Relation to Quantised Inertia}

McCulloch's Quantised Inertia (QI) framework~\cite{McCulloch2007} derives flat
rotation curves from vacuum mode truncation at the Hubble horizon.  Our work
is complementary:
\begin{itemize}
\item \textbf{QI} determines the \emph{total} vacuum energy (mode truncation
  at the outer boundary $\to$ rotation curve amplitude $\to$ MOND acceleration
  $a_0 \approx cH_0$).
\item \textbf{Efimov} determines the \emph{mode structure} (the $1/r^2$ halo
  reorganizes modes into a log-periodic spectrum $\to$ redshift quantization).
\end{itemize}
Both effects arise from the same vacuum modes interacting with the same
boundary conditions, but they address different observables.

\subsection{Asymmetry and Association Dependence}

Two key observational features of the Karlsson periodicity find natural
explanations in this framework.

\textbf{Positive-only excess.}  The intrinsic redshift is always positive (the
associated quasar is more redshifted than the parent galaxy, never
blueshifted).  This arises from the geometry of outward ejection through the
halo boundary: vacuum mode suppression is strongest near the parent and
relaxes outward, producing a net redward shift.  The Efimov spectrum provides
only positive eigenvalues---there is no mechanism for a blueshift counterpart.

\textbf{Association dependence.}  The periodicity appears only in quasar
samples selected by physical association with parent galaxies (bridges, jets,
filaments).  Field quasars show smooth redshift distributions.  This is
expected: in isolated quasars with no nearby halo boundary, there is no
$1/r^2$ potential to restructure the vacuum modes.  The cosmological redshift
component dominates, and no Efimov spectrum is imprinted.

\subsection{Predictions and Tests}

\begin{enumerate}
\item \textbf{Universality of the period}: Any system with an isothermal
  ($1/r^2$) mass profile should show log-periodic signatures with ratio
  $\alpha = 1.227$, regardless of the system's mass or size.  This could be
  tested in galaxy cluster halos.

\item \textbf{Absence in non-isothermal systems}: Systems with
  $\rho \propto r^{-\beta}$ for $\beta \neq 2$ would have different (or no)
  log-periodic structure.  The NFW profile ($\beta = 1$ at small~$r$,
  $\beta = 3$ at large~$r$) would predict weaker or absent periodicity
  outside the isothermal range.

\item \textbf{The $(2\pi)^2$ derivation}: The most important open problem is
  to rigorously derive the mode-counting factor.  This would elevate the
  result from a prediction to a derivation.
\end{enumerate}

\subsection{Why the Period Cannot Be Accidental}

The probability that $g = 24\pi^2$ accidentally matches Karlsson's period is
bounded by the Monte Carlo result ($p < 0.00002$).  But the argument is
stronger than this:
\begin{itemize}
\item The number 6 has independent geometric meaning (components of $h_{ij}$).
\item The factor $(2\pi)^2$ has independent geometric meaning (2D Fourier
  normalization).
\item Only $N_{\text{DOF}} = 6$ falls within $1\sigma$ of the observational
  value; the nearest competitors ($N = 5, 7$) are at ${\sim}\,1.5\sigma$.
\item The same $1/r^2$ profile that provides the Efimov potential is
  independently observed in galaxy halos.
\end{itemize}
Four independent coincidences aligning simultaneously is unlikely to be
accidental.


%======================================================================
\section{Conclusion}
\label{sec:conclusion}
%======================================================================

We have shown that the Karlsson redshift periodicity
$\Delta\log_{10}(1+z) = 0.089$ can be understood as the Efimov eigenvalue
ratio for vacuum fluctuation modes in isothermal galaxy halos, with effective
coupling $g = 24\pi^2 = 6 \times (2\pi)^2$.  The prediction matches
observation at $0.06\sigma$ with zero free parameters for the period.

The derivation chain rests on established observations (isothermal halos),
proven mathematics (the Efimov theorem), standard geometry (6~metric DOF), a
well-motivated hypothesis (all DOF fluctuate, gravitational Aharonov-Bohm),
and one not-yet-derived factor ($(2\pi)^2$ per DOF from horizon
mode-counting).

This is, to our knowledge, the first derivation of the Karlsson spacing from
physical principles.  Whether the remaining gap---the rigorous derivation of
$(2\pi)^2$---can be closed within the Jacobson thermodynamic framework or
requires new physics remains an open and important question.

\begin{acknowledgments}
All numerical results were verified using the reproduction script
\texttt{reproduce.py}, available at
\url{https://github.com/KeithBrodie/redshift-anomaly-paper}.
\end{acknowledgments}

\bibliography{references}

\end{document}
